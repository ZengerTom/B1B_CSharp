\chapter{Vorwort}
\section{\texttt{B1B} - Was ist das?}
Dieses Skript soll als Nachschlagewerk und Cheat Sheet dienen und eine Gedächtnisstütze zu einem bereits vertiefeten Thema bieten. Es soll keine Fachliteratur ersetzen und dient auch nicht zum Erlernen der Thematik, da auf ausführliche Erklärungen größtenteils verzichtet wird.
\section{Motivation}
Diese Idee zu diesem Kompendium enstand während meines Informatikstudiums. Da bereits erlernte Techniken, wie z.B. Programmier- oder Scriptsprachen, mit dem Fortschreiten des Studiums in den Hintergrund traten, zu einem späteren Zeitpunkt jedoch wieder benötigt wurden, war es unerlässlich sich diese wieder ins Gedächtnis zu rufen. Aus diesem Grund entstand dieses Werk als eine Art ``erweiterte Zusammenfassung''.
\section{Literatur und Grundlagen}
Folgende Werke fanden bei der Erstellung dieses Dokuments Beachtung. An dieser Stelle soll ausdrücklich erwähnt werden, dass sich diese Arbeit nicht als Plagiat oder Kopie genannter Literatur verstanden werden soll, sondern als Lernhilfe und Zusammenfassung.
\begin{itemize}
\item Thomas Theis (2015) - Einstieg in C\# mit Visual Studio 2015
\item Joseph Albahari, Ben Albahari (2016) - C\# 6.0 in a Nutshell
\item Joseph Albahari, Ben Albahari (2018) - C\# 7.0 - kurz \& gut
\item https://docs.microsoft.com/de-de/dotnet/csharp/
\end{itemize}
